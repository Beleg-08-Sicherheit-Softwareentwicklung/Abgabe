Die vorgestellten Modelle, Tools und Studien zeigten verschiedene Herangehensweisen und Umsetzungsmöglichkeiten des Shift-Left-Ansatzes. Jedes Konzept nutzte Shift-Left, um die Qualität und Sicherheit in Softwareprojekten zu stärken. Die daraus gewonnenen Erkenntnisse gilt es nun kritisch zu betrachten, um anschließend das Potenzial von Shift-Left realistisch einzuschätzen.

Es ist festzustellen, dass Shift-Left ein übergreifendes Konzept ist, das sich auf viele verschiedene Aspekte der Softwareentwicklung anwenden lässt \cite{andriadi_impact_2023}. Die Motivation für Shift-Left ergibt sich aus der Erkenntnis, dass späte Änderungen am Code teuer sind und ein frühes Beheben von Fehlern Kosten sowie Zeit spart. Diese Einsicht ist nicht neu und wurde bereits in den 1980er-Jahren von \citet{boehm_software_1984} ausführlich diskutiert. Die Idee, früh zu testen, ist damit älter als der Begriff Shift-Left selbst, der erst durch \cite{smith_shift-left_2001} geprägt wurde \cite{dawoud_better_2024}. Dennoch schmälert das nicht die Wirksamkeit des Konzepts.

Die vielfältige Anwendbarkeit von Shift-Left zeigt sich in den verschiedenen Formen dieses Ansatzes (\ref{sec:arten}). So kann Shift-Left sowohl auf Quellcode-Ebene genutzt werden, um den Entwicklungsprozess zu beeinflussen, als auch auf der Planungsebene, beispielsweise durch MBSLT. Eine detaillierte Anwendung dieser Konzepte liefert die Studie von \cite{andriadi_impact_2023}, die die Effektivität des Shift-Left-Ansatzes belegt. In dem untersuchten Projekt wurden Testing-Aktivitäten und Abläufe nach links verschoben sowie die Kommunikation zwischen Entwickler- und QA-Teams verbessert. Dies führte zu einer gesteigerten Softwarequalität und einer Reduzierung von Fehlern in der Anwendung. Durch Shift-Left konnte in diesem Fall die Qualität des Projekts gesichert werden.

Weiterhin zeigen Konzepte wie SAST, DAST und IAST, dass sich Sicherheitsüberprüfungen automatisieren und somit ebenfalls nach links verschieben lassen. Die Effektivität dieser Tools wurde anhand der Untersuchung von \citet{mateo_tudela_combining_2020} dargelegt. Durch die Nutzung von SAST, DAST und IAST können Sicherheitsprüfungen besser in agile Entwicklungsprozesse integriert werden, sodass Sicherheitslücken frühzeitig erkannt und behoben werden können.

Beispiele für eine strukturierte Anwendung von Shift-Left liefern die Frameworks Scrum4Safety und S2C-SAFe \cite{barbareschi_scrum_2022} \cite{moyon_how_2020}. Diese bieten einen strukturierten Ansatz für die agile Entwicklung sicherheitskritischer Software und kombinieren somit Agilität und Sicherheitsanforderungen effektiv. In beiden Modellen ist Shift-Left in verschiedenen Bereichen erkennbar: Neben dem frühzeitigen Testing auf Code-Ebene werden auch abstrakte Elemente wie Safety Stories eingeführt, um Sicherheitsanforderungen bereits vor der Entwicklung zu festigen.

Die Modelle und Tools zeigen, dass Shift-Left ein sinnvoller und effektiver Ansatz zur Verbesserung der Softwarequalität und -sicherheit ist. Allerdings kommt Shift-Left-Security nicht ohne Herausforderungen. Beispielsweise stellten \citet{andriadi_impact_2023} fest, dass durch die frühzeitige Integration von Sicherheitstests die Entwickler mehr Zeit benötigten und eine intensivere Kommunikation zwischen Entwickler- und QA-Teams erforderlich war. Dies erhöht sowohl die Kosten als auch die Komplexität der Umsetzung.

Ähnliches lässt sich in den Frameworks Scrum4Safety und S2C-SAFe beobachten. Im Vergleich zu rein agilen Modellen erfordern sie stärker strukturierte Prozesse und eine klarere Rollenverteilung. Für Teams, die vollständig agil arbeiten, kann dies einschränkend wirken. Zwar ermöglichen die Frameworks eine Kombination von Agilität und Sicherheit, jedoch nicht nahtlos und ohne zusätzliche Aufwände. Der verstärkte Fokus auf Sicherheit führt zu einer gewissen Einschränkung des Arbeitsprozesses und erfordert geschulte Entwickler.

Das zeigt, dass Shift-Left-Security zwar ein sinnvoller und effektiver Ansatz ist, aber keine Universallösung darstellt. Der Schlüssel liegt in einer sorgfältigen Implementierung und einer ausführlichen Analyse im Vorfeld. Jedes sicherheitskritische Projekt sollte gezielt geplant werden, um zu bestimmen, wie Shift-Left optimal integriert werden kann. Dafür können etablierte Frameworks wie Scrum4Safety oder S2C-SAFe genutzt werden – vorausgesetzt, es ist bekannt, dass sie für den jeweiligen Kontext eine sinnvolle Lösung darstellen.